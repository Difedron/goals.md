\documentclass[12pt]{article}
 \usepackage{amsmath}
 \usepackage{latexsym}
 \usepackage{amsfonts}
 \usepackage[normalem]{ulem}
 \usepackage{soul}
 \usepackage{array}
 \usepackage{amssymb}
 \usepackage{extarrows}
 \usepackage{graphicx}

 \usepackage[backend=biber,
 style=numeric,
 sorting=none,
 isbn=false,
 doi=false,
 url=false,
 ]{biblatex}\addbibresource{bibliography.bib}

\usepackage{listings}
\usepackage{xcolor} % для цветов

\lstset{
  language=Python,            % язык (Python, C, Java...)
  basicstyle=\ttfamily\small, % шрифт кода
  numbers=left,               % нумерация строк
  numberstyle=\tiny\color{gray},
  backgroundcolor=\color{lightgray!20},
  frame=single,                % рамка вокруг кода
  keywordstyle=\color{blue},   % ключевые слова
  commentstyle=\color{green!50!black}, % комментарии
  stringstyle=\color{orange},
  showstringspaces=false
}


  \usepackage{subfig}
 \usepackage{wrapfig}
 
 \usepackage{wasysym}
 \usepackage{enumitem}
 \usepackage{adjustbox}
 \usepackage{ragged2e}
 \usepackage[svgnames,table]{xcolor}
 \usepackage{tikz}
 \usepackage{longtable}
 \usepackage{changepage}
 \usepackage{setspace}
 \usepackage{hhline}
 \usepackage{multicol}
 \usepackage{tabto}
 \usepackage{float}
 \usepackage{multirow}
 \usepackage{makecell}
 \usepackage{fancyhdr}
 \usepackage[toc,page]{appendix}
 \usepackage[colorlinks=true, linkcolor=blue, urlcolor=blue, citecolor=blue]{hyperref}
 \usetikzlibrary{shapes.symbols,shapes.geometric,shadows,arrows.meta}
 \tikzset{>={Latex[width=1.5mm,length=2mm]}}
 \usepackage{flowchart}\usepackage[paperheight=11.69in,paperwidth=8.27in,left=1.18in,right=0.39in,top=0.79in,bottom=0.79in,headheight=1in]{geometry}
 \usepackage[utf8]{inputenc}
 \usepackage[russian]{babel}
 \usepackage[T1,T2A]{fontenc}
 \TabPositions{0.5in,1.0in,1.5in,2.0in,2.5in,3.0in,3.5in,4.0in,4.5in,5.0in,5.5in,6.0in,6.5in,}

  \urlstyle{same}

 

 
  \setcounter{tocdepth}{5}
 \setcounter{secnumdepth}{5}

 
  \setlistdepth{9}
 \renewlist{enumerate}{enumerate}{9}
 		\setlist[enumerate,1]{label=\arabic*)}
 		\setlist[enumerate,2]{label=\alph*)}
 		\setlist[enumerate,3]{label=(\roman*)}
 		\setlist[enumerate,4]{label=(\arabic*)}
 		\setlist[enumerate,5]{label=(\Alph*)}
 		\setlist[enumerate,6]{label=(\Roman*)}
 		\setlist[enumerate,7]{label=\arabic*}
 		\setlist[enumerate,8]{label=\alph*}
 		\setlist[enumerate,9]{label=\roman*}

  \renewlist{itemize}{itemize}{9}
 		\setlist[itemize]{label=$\cdot$}
 		\setlist[itemize,1]{label=\textbullet}
 		\setlist[itemize,2]{label=$\circ$}
 		\setlist[itemize,3]{label=$\ast$}
 		\setlist[itemize,4]{label=$\dagger$}
 		\setlist[itemize,5]{label=$\triangleright$}
 		\setlist[itemize,6]{label=$\bigstar$}
 		\setlist[itemize,7]{label=$\blacklozenge$}
 		\setlist[itemize,8]{label=$\prime$}

  \pagenumbering{gobble}
 \setlength{\topsep}{0pt}\setlength{\parskip}{9.96pt}
 \setlength{\parindent}{0pt}

   %%%%%%%%%%%%  This sets linespacing (verticle gap between Lines) Default=1 %%%%%%%%%%%%%%

 
  \renewcommand{\arraystretch}{1.3}

 
  %%%%%%%%%%%%%%%%%%%% Document code starts here %%%%%%%%%%%%%%%%%%%%

 
 
  \begin{document}
 \begin{Center}
 ИТМО
 \end{Center}\par

  \begin{Center}
 Институт математики
 \end{Center}\par
 \vspace{\baselineskip}
 ОТЧЕТ \\
 ЗАЩИЩЕН С ОЦЕНКОЙ\par

  ПРЕПОДАВАТЕЛЬ\par

 
 
  %%%%%%%%%%%%%%%%%%%% Table No: 1 starts here %%%%%%%%%%%%%%%%%%%%

 
  \begin{table}[H]
  			\centering
 \begin{tabular}{p{2.05in}p{0.0in}p{1.76in}p{-0.01in}p{1.89in}}
 %row no:1
 \multicolumn{1}{p{2.05in}}{\Centering {проф., д.т.н., проф.}} & 
 \multicolumn{1}{p{0.0in}}{} & 
 \multicolumn{1}{p{1.76in}}{} & 
 \multicolumn{1}{p{-0.01in}}{} & 
 \multicolumn{1}{p{1.89in}}{\Centering {Усачева Д.М.}} \\
 \hhline{-~-~-}
 %row no:2
 \multicolumn{1}{p{2.05in}} {\Centering{\fontsize{10pt}{12.0pt}\selectfont  {должность, уч. степень, звание}}} & 
 \multicolumn{1}{p{0.0in}}{} & 
 \multicolumn{1}{p{1.76in}} {\Centering{\fontsize{10pt}{12.0pt}\selectfont  {подпись, дата}}} & 
 \multicolumn{1}{p{-0.01in}}{} & 
 \multicolumn{1}{p{1.89in}} {\Centering{\fontsize{10pt}{12.0pt}\selectfont  {инициалы, фамилия}}} \\
 \hhline{~~~~~}

  \end{tabular}
  \end{table}

 
  %%%%%%%%%%%%%%%%%%%% Table No: 1 ends here %%%%%%%%%%%%%%%%%%%%

 
  \vspace{\baselineskip}

 
  %%%%%%%%%%%%%%%%%%%% Table No: 2 starts here %%%%%%%%%%%%%%%%%%%%

 
  \begin{table}[H]
  			\centering
 \begin{tabular}{p{6.49in}}
 %row no:1
 \multicolumn{1}{p{6.49in}}{\fontsize{14pt}{16.8pt}\selectfont {\section*{\Centering {ОТЧЕТ О ЛАБОРАТОРНОЙ РАБОТЕ №0}}}}\\
 \hhline{~}
 %row no:2
 \multicolumn{1}{p{6.49in}}{\section*{\Centering {Вводная}}
 } \\
 \hhline{~}
 %row no:3
 \multicolumn{1}{p{6.49in}}{\subsubsection*{\Centering {по курсу: МАШИННОЕ ОБУЧЕНИЕ}}
 } \\
 \hhline{~}
 %row no:4
 \multicolumn{1}{p{6.49in}}{} \\
 \hhline{~}
 %row no:5
 \multicolumn{1}{p{6.49in}}{} \\
 \hhline{~}

  \end{tabular}
  \end{table}

 
  %%%%%%%%%%%%%%%%%%%% Table No: 2 ends here %%%%%%%%%%%%%%%%%%%%

  РАБОТУ ВЫПОЛНИЛ\par

 
 
  %%%%%%%%%%%%%%%%%%%% Table No: 3 starts here %%%%%%%%%%%%%%%%%%%%

 
  \begin{table}[H]
  			\centering
 \begin{tabular}{p{1.3in}p{1.0in}p{-0.04in}p{1.63in}p{-0.04in}p{1.63in}}
 %row no:1
 \multicolumn{1}{p{1.3in}}{СТУДЕНТ ГР. №} & 
 \multicolumn{1}{p{1.0in}}{\Centering {R3340}} & 
 \multicolumn{1}{p{-0.04in}}{} & 
 \multicolumn{1}{p{1.63in}}{} & 
 \multicolumn{1}{p{-0.04in}}{} & 
 \multicolumn{1}{p{1.63in}}{\Centering {Холухина Д.Е.}} \\
 \hhline{~-~-~-}
 %row no:2
 \multicolumn{1}{p{1.3in}}{} & 
 \multicolumn{1}{p{1.0in}}{} & 
 \multicolumn{1}{p{-0.04in}}{} & 
 \multicolumn{1}{p{1.63in}} {\Centering{\fontsize{10pt}{12.0pt}\selectfont  {подпись, дата}}} & 
 \multicolumn{1}{p{-0.04in}}{} & 
 \multicolumn{1}{p{1.63in}} {\Centering{\fontsize{10pt}{12.0pt}\selectfont  {инициалы, фамилия}}} \\
 \hhline{~~~~~~}

  \end{tabular}
  \end{table}

 
  %%%%%%%%%%%%%%%%%%%% Table No: 3 ends here %%%%%%%%%%%%%%%%%%%%

 
  \vspace{\baselineskip}
 \vspace{\baselineskip}
 \vspace{\baselineskip}
 \vspace{\baselineskip}
 \vspace{\baselineskip}
\vspace{\baselineskip}
 \vspace{\baselineskip}
\vspace{\baselineskip}
\vspace{\baselineskip}
\vspace{\baselineskip} \vspace{\baselineskip}
\vspace{\baselineskip}
\vspace{\baselineskip}


 \begin{Center}
 Санкт-Петербург 
 \par \the\year{}
 \end{Center}\par


\newpage

\setcounter{page}{2}
\tableofcontents

\newpage



\section*{Задание 1.}
\addcontentsline{toc}{section}{Задание 1.}

Целью первого задания было поразмышлять о своих целях и задачах на текущий семестр.
С помощью веб-интерфейса GitHub создали файл goals.md в своем репозитории.
Используя язык разметки markdown, создали заголовки "Мой опыт" и "Мои цели".
Написали один абзац (1-5 предложений) о своем опыте по разработке программного обеспечения. Добавили работающую ссылку в написанный текст (на мой профиль в GitHub).
Создали нумерованный список из 2 или 3 целей, которых планируем достичь в рамках данного курса.
Создайли ненумерованный список из 3-9 задач, над которыми собирались работать в рамках данного курса для достижения поставленных целей.
Создали папку resources в корне репозитория. Загрузили в эту папку изображение и добавили его в наш .md файл.
Так же не забыли закоммитить файлы в свой репозиторий на GitHub.
Проверили, что созданный .md файл отформатирован и отображается при предпросмотре именно так, как планировалось.

\section*{Задание 2.}
\addcontentsline{toc}{section}{Задание 2.}
Целью второго задания было научиться использовать git из командной строки.

Мы ознакомились с инструкцией по работе с git и github из командной строки.
Клонировали свой репозиторий на локальный компьютер используя команду git clone (с ссылкой на свой репозиторий) из командной строки.
Создали новый файл info.md в своей локальной копии репозитория.
Записали в этот файл свое ФИО, название используемой операционной системы, текущие дату и время, - каждый элемент с новой строки (всего получится 3 строки).
Выполнили команды git add info.md и git commit в командной строке.
Ввели краткое описание коммита.
Выполнили команду git push.
Создали папку data в корне репозитория и в ней файл dataset.csv. Создали файл .gitignore, в который добавили строку data/. Далее закоммитили .gitignore и попробовали закоммитить data/dataset.csv, после чего отправили коммиты в репозиторий на github.com с помщью git push. 


\textbf{Эксперимент с файлом .gitignore}
\addcontentsline{toc}{subsection}{Эксперимент с файлом .gitignore}


В корне репозитория была создана папка \texttt{data} и в ней файл \texttt{dataset.csv}.
Затем в корне репозитория был создан файл \texttt{.gitignore}, в который добавлена строка:
\begin{lstlisting}
data/ 
\end{lstlisting}
Данная строка означает, что система контроля версий Git должна игнорировать
все файлы и подпапки внутри каталога \texttt{data}.

После этого файл \texttt{.gitignore} был добавлен в индекс и закоммичен:
\begin{lstlisting}
git add .gitignore
git commit -m "Добавлен .gitignore для игнора папки data"
\end{lstlisting}

Затем была предпринята попытка добавить файл \texttt{data/dataset.csv}:
\begin{lstlisting}
git add data/dataset.csv
\end{lstlisting}
\newpage
Git выдал сообщение:
\begin{lstlisting}
The following paths are ignored by one of your .gitignore files:
data/dataset.csv
hint: Use -f if you really want to add them.
\end{lstlisting}
Это означает, что путь \texttt{data/dataset.csv} игнорируется в соответствии
с настройками \texttt{.gitignore}, поэтому Git не добавляет этот файл в индекс
и, следовательно, не может его закоммитить.

Далее был выполнен пуш изменений:
\begin{lstlisting}
git push
\end{lstlisting}

В удалённый репозиторий на GitHub был отправлен только коммит с файлом
\texttt{.gitignore}, тогда как папка \texttt{data} и файл \texttt{dataset.csv}
в репозитории отсутствуют.


\newpage

\section*{Приложение.}
\addcontentsline{toc}{section}{Приложение.}
\textbf{Содержимое файла goals.md:}

# Мой опыт
Когда-то я пользовалась GitHubом в Яндекс лицее, но уже ничег не помню, вот ссылка:
(https://github.com/Difedron)


# Мои цели

В рамках данного курса я планирую:
1. Освежить свои знания ML
2. Научиться нормально пользоваться GitHubом
3. Научиться обучать крутые модели

Чтобы этого достичь, я буду:
- Слушать лекции
- Делать практические задания
- Хорошо высыпаться и кушать
- Усердно трудиться



\textbf{Содержимое файла info.md:}

Холухина Диана Евгеньевна

macOS Sonoma 14.3

2025-09-19 16:38:50


\textbf{Содержимое файла .gitignore:}
\par

data/






\end{document}